% Created 2016-06-30 四 20:08

\documentclass{article}
\usepackage[
a4paper,
headheight=12pt,
paperwidth=210mm,
headsep=20pt,
includeheadfoot,
centering,
top=0.5cm,
bottom=1cm,
left=3cm
]{geometry}

%\Usepackage{pdflatex}


\usepackage{xeCJK}  %

%\usepackage{fontspec}
%\setromanfont{Microsoft Sans Serif}
\usepackage{fontspec,xunicode,xltxtra}
\setmainfont{Times New Roman}
\setsansfont{Times New Roman}                                                       
\setmonofont{Times}                                                     
%\newcommand\fontnamemono{STXihei}                                            
\newfontinstance\MONO{\fontnamemono}                                        
\newcommand{\mono}[1]{{\MONO #1}}                                           
\setCJKmainfont[BoldFont=STHeiti, ItalicFont=STKaiti]{STSong}%中文字体        
\setCJKsansfont{STHeiti}                                                      
\setCJKmonofont{STFangsong} 


\newcommand\fontnamehei{STHeiti}
\newcommand\fontnamesong{STSong}
\newcommand\fontnamekai{STKaiti}
\newcommand\fontnamefangsong{STFangsong}
\setCJKfamilyfont{kai}{\fontnamekai}
\setCJKfamilyfont{hei}{\fontnamehei}
\setCJKfamilyfont{song}{\fontnamesong}
\setCJKfamilyfont{fs}{\fontnamefangsong}

\newcommand{\song}{\CJKfamily{song}}    % 宋体
\newcommand{\fs}{\CJKfamily{fs}}             % 仿宋体
\newcommand{\kai}{\CJKfamily{kai}}          % 楷体
\newcommand{\hei}{\CJKfamily{hei}}         % 黑体


\newcommand{\yihao}{\fontsize{26pt}{36pt}\selectfont}           % 一号, 1.4 倍行距
\newcommand{\erhao}{\fontsize{22pt}{28pt}\selectfont}          % 二号, 1.25倍行距
\newcommand{\xiaoer}{\fontsize{18pt}{18pt}\selectfont}          % 小二, 单倍行距
\newcommand{\sanhao}{\fontsize{16pt}{24pt}\selectfont}        % 三号, 1.5倍行距
\newcommand{\xiaosan}{\fontsize{15pt}{22pt}\selectfont}        % 小三, 1.5倍行距
\newcommand{\sihao}{\fontsize{14pt}{21pt}\selectfont}            % 四号, 1.5 倍行距
\newcommand{\banxiaosi}{\fontsize{13pt}{19.5pt}\selectfont}    % 半小四, 1.5倍行距
\newcommand{\xiaosi}{\fontsize{12pt}{18pt}\selectfont}            % 小四, 1.5倍行距
\newcommand{\dawuhao}{\fontsize{11pt}{11pt}\selectfont}       % 大五号, 单倍行距
\newcommand{\wuhao}{\fontsize{10.5pt}{15.75pt}\selectfont}    % 五号, 单倍行距

\usepackage{etoolbox}  % Quote 部份的字型設定
\newfontfamily\quotefont{Songti SC}
\AtBeginEnvironment{quote}{\quotefont\small}

\XeTeXlinebreaklocale ``zh''
\XeTeXlinebreakskip = 0pt plus 1pt
\linespread{1.391}
%首行缩进
\usepackage{indentfirst}
\setlength{\parindent}{0.75cm}
%\addtolength{\parskip}{3pt}

\parskip=0bp plus 10bp minus 1bp %段距为最大10磅,仅为行距一半,最小可以压缩1磅。
\raggedbottom

%设置调用图表宏包
\usepackage[format=hang,labelsep=space]{caption}
\intextsep=6bp %段距为最大10磅,仅为行距一半,最小可以压缩1磅。
\textfloatsep=6bp %设置浮动体在页面顶端或底端时多个之间的距离
\floatsep=6bp %分别设置表和图的标题与正文的距离
\captionsetup[figure]{aboveskip=0bp,belowskip=0bp}
\captionsetup[table]{aboveskip=0bp,belowskip=6bp}

%设置图表说明
\usepackage{booktabs,tabularx,threeparttable,longtable}
%\renewcommand{\thefigure}{(+ 0 org-level-1)\textendash\arabic{figure}}
%\renewcommand{\thetable}{'org-level-1\textendash\arabic{table}}
%
\usepackage[below]{placeins}
\usepackage{flafter}

%\usepackage{indentfirst}
%\setlength{\parindent}{2em}


% [FIXME] ox-latex 的設計不良導致 hypersetup 必須在這裡插入
\usepackage{hyperref}
\hypersetup{
  colorlinks=true, %把紅框框移掉改用字體顏色不同來顯示連結
  linkcolor=blue, %[rgb]{0,0.37,0.53},
  citecolor=magenta, %[rgb]{0,0.47,0.68},
  filecolor=yellow, %[rgb]{0,0.37,0.53},
  urlcolor=red, %[rgb]{0,0.37,0.53},
  pagebackref=true,
  linktoc=all,}

\usepackage{hyperref}
\usepackage[utf8]{inputenc}
\usepackage{fixltx2e}
\usepackage{graphicx}
\usepackage{calc}
\usepackage{longtable}
\usepackage{float}
\usepackage{wrapfig}
\usepackage{rotating}
\usepackage[normalem]{ulem}
\usepackage{amsmath}
\usepackage{textcomp}
\usepackage{marvosym}
\usepackage{wasysym}
\usepackage{adjmulticol}
\usepackage{amssymb}
\tolerance=1000
\author{姓名:刘福雁\thanks{liufuyan15@mails.ucas.ac.cn}}
\date{学号:201518000907060}
\title{对机器学习及人工智能的思考}
\hypersetup{
  pdfkeywords={},
  pdfsubject={},
  pdfcreator={Emacs 24.5.1 (Org mode 8.2.10)}}
\begin{document}

\maketitle
\kai\xiaosi\noindent 注:笔者是一位非计算机专业(凝聚态物理)的学生,本文是笔者在阅读了一些机器学习方面的资料后,整理的一些读书笔记。

\song\banxiaosi  大家都知道,在今 年 3 月,在韩国首尔举行的谷歌 DeepMind 围棋挑战赛,人 工智能围棋软件 ALphaGo 以 4:1 战胜了 韩国棋手李世乭九段。无论是支持李世乭的看客还是抱着好奇态度支持ALphaGo的旁观者都大吃一惊。以至于近半年来关 于人工智能和机器学习的话题迅速升 温,引起了社会各界的关心。ALphaGo是机器学习的一个案例。而机器学习是人工智能的一个分支学科,主要研究的是让机器从过去的经 历中学习经验,对数据的不确定性进行 建模,在未来进行预测。在机器学习领域,通常有四种算法:有监督学习、半监督学习、无监督学习和增强学习。这里提到的有没有“监督”指的是机器的学习阶段,能否看到被提供样本的标签特征,这些标签特征可以认为是人类本来就知道的提供的这些样本的属性。

除了前面提到的四 大类算法,机器学习还 有很多有趣的子领域。在大数据时代,机器学习领域有 一个热点是把系统与算法结合,设计大 规模分布式的机器学习算法与系统,使 得机器学习算法可以在多处理器和多 机器的集群环境下作业,处理更大量级 的数据。笔者认为:未来的物理实验会得到越来越多的数据,最终我们探测这个世界以及宇宙只会得到大量有用或者无用的数据,而计算机发展的速度远远不及数据膨胀的速度,也就是说,即便我们拥有大自然给我们的启迪————“数据”,我们也无法揭开其中的奥秘。因此未来的世界我们需要更快处理速度的计算机。已然,目前为止,计算机不在是一个独立的学科,在 大数据的背景下,计算机科学正逐渐演变 成一个越来越强调跨领域合作的学 科。如何有效地把系统和机器学习方 法相结合来处理海量数据,这将是未来 人工智能和计算机科学发展的关键。

人工智能的发展时如此迅速,我们今天利用人工智能可以完成很多工作。甚至可以说,没有人工智能,我们寸步难行。但是自打人工智能的产品问世以来,对人工智能担忧的声音从来不绝于耳。我们应该走向哪里,人工智能又会发生什么?笔者读到的一个很有趣的回答是:美国乔治亚理工大学机器学习助理教 授Le Song认为这种论断具有基本的逻 辑错误,因为古今中外,会下围棋并不 代表能统治世界,绝大多数的国家领导 人也都不是围棋棋手。从事实出发谷 歌的 ALphaGo 系统仅仅是在人类棋谱 上训练而成的计算机围棋软件系统,其 根本不具有理解人类语言、图片和推理 等其他机器智能。并且ALphaGo的背后是几十名谷歌工程师的辛苦工作,涉及算法和调试程序的反复迭代,才取得的成果。也就是说ALphaGo本身是科学家智慧的结晶。

当然,这种回复是不具有说服力的,这只是从基本的语言逻辑上做了分析。但是可能未来的人工智能产品可以掌握人类的某些思想特征,从而具有一定的推理,判断,预测的思考及学习能力,那我们又该怎么管理这些人工智能呢。因此,笔者认为对待人工智能人类应强调其利于人类的优点,而抑制其违背人类思想的缺点。但是我们不得不承认,这些担心目前来说有些多余,我们面临的实际情况是:对于自然语言理解,虽然经过了数十年的发 展,依然没有人工智能系统可以做到完 全正确地理解人类的语言;在机器人领域,即使工 业机器人发展迅速,我们依然没有看到 具有常识和推理能力的智能家庭机器 人;在计算机视觉领域,即使我们在人 脸识别和图片分类上取得了不小的成 就,但是对于关系理解和完整的场景认 知,现在系统能做到的还很有限。对人工智能的开发,何尝不是人类对自己本身的认识和研究,而人类作为大自然的一部分,又是一环扣一环缠绕在一起的,也许我们对自身的研究需要各个学科的综合应用,如自然科学提供的自然基本原理,神经科学提供的基本的神经模式,心理学提供的人类对外界环境的基本反应模式等。所以机器学习或者人工智能不是单纯的一个独立的学科,是一门综合性学科,就现在的这些基本学科发展而言,没有任何一门学科已经对本领域的内容研究地很透彻(包括物理学),更别说是交叉学科了。

总而言之,现代社会是人工智能发展的 一个令人兴奋的时期,机器学习技术对 于整个人类的发展,也是具有不可估量 的潜力。我们应该继续完善各个学科以及交叉学科的发展,并且正视科学技术发展 的进步,理性看待利用科学取得的结果。
% Emacs 24.5.1 (Org mode 8.2.10)
\end{document}
