% Created 2016-07-07 四 17:36

\documentclass{article}
\usepackage[
a4paper,
headheight=12pt,
paperwidth=210mm,
headsep=20pt,
includeheadfoot,
centering,
top=0cm,
bottom=0.5cm,
left=0.4in,
right=0.4in
]{geometry}
\usepackage{xeCJK}  %
\usepackage{fontspec,xunicode,xltxtra}
\setmainfont{Times New Roman}
\setsansfont{Times New Roman}
\setmonofont{Times}                                                     
%\newcommand\fontnamemono{STXihei}
%\newfontinstance\MONO{\fontnamemono}                                        
\newcommand{\mono}[1]{{\MONO #1}}
\setCJKmainfont[BoldFont=STHeiti, ItalicFont=STKaiti]{STSong}%中文字体
\setCJKsansfont{STHeiti}
\setCJKmonofont{STFangsong} 
%设置字体快捷命令 
\newcommand\fontnamehei{STHeiti}
\newcommand\fontnamesong{STSong}
\newcommand\fontnamekai{STKaiti}
\newcommand\fontnamefangsong{STFangsong}
\newcommand\fontnamehuahei{STHeiti}
\setCJKfamilyfont{kai}{\fontnamekai}
\setCJKfamilyfont{hei}{\fontnamehei}
\setCJKfamilyfont{song}{\fontnamesong}
\setCJKfamilyfont{fs}{\fontnamefangsong}
\setCJKfamilyfont{huahei}{\fontnamehuahei}
\newcommand{\song}{\CJKfamily{song}}    % 宋体
\newcommand{\fs}{\CJKfamily{fs}}             % 仿宋体
\newcommand{\kai}{\CJKfamily{kai}}          % 楷体
\newcommand{\hei}{\CJKfamily{hei}}         % 黑体
\newcommand{\hhei}{\CJKfamily{huahei}}       %娃娃体
%设置字号快捷命令
\newcommand{\yihao}{\fontsize{26pt}{36pt}\selectfont}           % 一号, 1.4 倍行距
\newcommand{\erhao}{\fontsize{22pt}{28pt}\selectfont}          % 二号, 1.25倍行距
\newcommand{\xiaoer}{\fontsize{18pt}{18pt}\selectfont}          % 小二, 单倍行距
\newcommand{\sanhao}{\fontsize{16pt}{24pt}\selectfont}        % 三号, 1.5倍行距
\newcommand{\xiaosan}{\fontsize{15pt}{22pt}\selectfont}        % 小三, 1.5倍行距
\newcommand{\sihao}{\fontsize{14pt}{21pt}\selectfont}            % 四号, 1.5 倍行距
\newcommand{\banxiaosi}{\fontsize{13pt}{19.5pt}\selectfont}    % 半小四, 1.5倍行距
\newcommand{\xiaosi}{\fontsize{12pt}{18pt}\selectfont}            % 小四, 1.5倍行距
\newcommand{\dawuhao}{\fontsize{11pt}{11pt}\selectfont}       % 大五号, 单倍行距
\newcommand{\wuhao}{\fontsize{10.5pt}{15.75pt}\selectfont}    % 五号, 单倍行距

\usepackage{etoolbox}  % Quote 部份的字型設定
\newfontfamily\quotefont{Songti SC}
\AtBeginEnvironment{quote}{\quotefont\small}

\XeTeXlinebreaklocale ``zh''
\XeTeXlinebreakskip = 0pt plus 1pt
\linespread{1.391}
%首行缩进
\usepackage{indentfirst}
\setlength{\parindent}{0.75cm}
\parskip=0bp plus 10bp minus 1bp %段距为最大10磅,仅为行距一半,最小可以压缩1磅。
\raggedbottom

%设置调用图表宏包
\usepackage[format=hang,labelsep=space]{caption}
\intextsep=6bp %段距为最大10磅,仅为行距一半,最小可以压缩1磅。
\textfloatsep=6bp %设置浮动体在页面顶端或底端时多个之间的距离
\floatsep=6bp %分别设置表和图的标题与正文的距离
\captionsetup[figure]{aboveskip=0bp,belowskip=0bp}
\captionsetup[table]{aboveskip=0bp,belowskip=6bp}

%设置图表说明
\usepackage{booktabs,tabularx,threeparttable,longtable}
%\captionsetup{font=capfont} %生成tex文件不能编译
%\renewcommand{\thefigure}{(+ 0 org-level-1)\textendash\arabic{figure}}
%\renewcommand{\thetable}{'org-level-1\textendash\arabic{table}}

\usepackage[below]{placeins}
\usepackage{flafter}

% [FIXME] ox-latex 的設計不良導致 hypersetup 必須在這裡插入
\usepackage{hyperref}
\hypersetup{
  colorlinks=true, %把紅框框移掉改用字體顏色不同來顯示連結
  linkcolor=blue, %[rgb]{0,0.37,0.53},
  citecolor=magenta, %[rgb]{0,0.47,0.68},
  filecolor=yellow, %[rgb]{0,0.37,0.53},
  urlcolor=red, %[rgb]{0,0.37,0.53},
  pagebackref=true,
  linktoc=all,}

\usepackage{hyperref}
\usepackage[utf8]{inputenc}
\usepackage{fixltx2e}
\usepackage{graphicx}
\usepackage{calc}
\usepackage{longtable}
\usepackage{float}
\usepackage{wrapfig}
\usepackage{rotating}
\usepackage[normalem]{ulem}
\usepackage{amsmath}
\usepackage{textcomp}
\usepackage{marvosym}
\usepackage{wasysym}
\usepackage{adjmulticol}
\usepackage{amssymb}
\usepackage{enumerate}
\usepackage{multicol}
\tolerance=1000
\begin{document}
%\title{\Large\bf Emacs快捷键列表}
%%\author{{刘福雁}\thanks{liufuyan15@mails.ucas.ac.cn}}
%\maketitle
\begin{center}
\hei\sihao{Emacs快捷键列表}
\end{center}
\noindent \textbf{注:}本文在署名-非商业性使用-捆同方式共享  3.0版权协议下发布,转载请注明出自aifreedom.com

\begin{multicols}{2}

\noindent C = Control \\
M = Meta = Alt  \\
Esc Del = Backspace 

\noindent \textbf{\xiaosi 基本快捷键(Basic)}
\vspace{-9pt} 
\begin{verbatim}
C-x C-f "find"文件, 即在缓冲区打开/新建一个文件
C-x C-s 保存文件
C-x C-w 使用其他文件名另存为文件
C-x C-v 关闭当前缓冲区文件并打开新文件
C-x i 在当前光标处插入文件
C-x b 新建/切换缓冲区 
C-x C-b 显示缓冲区列表 
C-x k 关闭当前缓冲区
C-z 挂起emacs
C-x C-c 关闭emacs
\end{verbatim}
\vspace{-9pt}
\noindent \textbf{\xiaosi 光标移动基本快捷键(Basic Movement)}
\vspace{-9pt} 
\begin{verbatim}
C-f 后一个字符 
C-b 前一个字符 
C-p 上一行
C-n 下一行
M-f 后一个单词 
M-b 前一个单词 
C-a 行首
C-e 行尾
C-v 向下翻一页 
M-v 向上翻一页 
M-< 到文件开头 
M-> 到文件末尾
\end{verbatim}
\vspace{-9pt}
\noindent \textbf{\xiaosi 编辑(Editint)}
\vspace{-9pt} 
\begin{verbatim}
M-n 重复执行后一个命令n次
C-u 重复执行后一个命令4次
C-u n 重复执行后一个命令n次
C-d 删除(delete)后一个字符
M-d 删除后一个单词
Del 删除前一个字符
M-Del 删除前一个单词
C-k 移除(kill)一行
C-Space 设置开始标记(例如标记区域)
C-@ 功能同上,用于C-Space被操作系统拦截的情况
C-w 移除(kill)标记区域的内容
M-w 复制标记区域的内容
C-y  召回(yank)复制/移除的区域/行
M-y 召回更早的内容(在kill缓冲区内循环)
C-x C-x 交换光标和标记
C-t 交换两个字符的位置 
M-t 交换两个单词的位置 
C-x C-t 交换两行的位置
M-u 使从光标位置到单词结尾处的字母变成大写
M-l 与M-u相反
M-c 使从光标位置开始的单词的首字母变为大写
\end{verbatim}
\vspace{-9pt}
\noindent \textbf{\xiaosi 重要快捷键(Important)}
\vspace{-9pt}
\begin{verbatim}
C-g 停止当前运行/输入的命令
C-x u 撤销前一个命令
M-x revert-buffer RETURN(照着这个输入)撤销上次存盘后所有改动
M-x recover-file RETURN 从自动存盘文件恢复
M-x recover-session RETURN 如果你编辑了几个文件, 用这个恢复
\end{verbatim}
\vspace{-9pt}
\noindent \textbf{\xiaosi 在线帮助(Online-Help)}
\vspace{-9pt}
\begin{verbatim}
C-h c 显示快捷键绑定的命令
C-h k 显示快捷键绑定的命令和它的作用
C-h l 显示最后100个键入的内容
C-h w 显示命令被绑定到哪些快捷键上
C-h f 显示函数的功能
C-h v 显示变量的含义和值
C-h b 显示当前缓冲区所有可用的快捷键
C-h t 打开emacs教程
C-h i 打开info阅读器
C-h C-f 显示emacs FAQ

C-h p 显示本机Elisp包的信息
搜索/替换(Seach/Replace)
C-s 向后搜索
C-r 向前搜索
C-g 回到搜索开始前的位置(如果你仍然在搜索模式中)
M-% 询问并替换(query replace)
Space或y  替换当前匹配
Del或n 不要替换当前匹配
.  仅仅替换当前匹配并退出(替换)
,  替换并暂停(按Space或y继续)
! 替换以下所有匹配
^ 回到上一个匹配位置
RETURN或q 退出替换
使用正则表达式(Regular expression)搜索/替换
可在正则表达式中使用的符号:
^ 行首
$ 行尾
. 单个字符
.* 任意多个(包括没有)字符
\< 单词开头
\> 单词结尾
[] 括号中的任意一个字符(例如[a-z]表示所有的小写字母)

M C-s RETURN 使用正则表达式向后搜索 
M C-r RETURN 使用正则表达式向前搜索 
C-s 增量搜索
C-s 重复增量搜索
C-r 向前增量搜索
C-r 重复向前增量搜索
M-x query-replace-regexp 使用正则表达式搜索并替换
\end{verbatim}
\vspace{-9pt}
\noindent \textbf{\xiaosi 窗口命令(Window Commands)}
\vspace{-9pt}
\begin{verbatim}
C-x 2 水平分割窗格
C-x 3 垂直分割窗格
C-x o 切换至其他窗格
C-x 0 关闭窗格

C-x 1 关闭除了光标所在窗格外所有窗格
C-x ^ 扩大窗格
M-x shrink-window 缩小窗格
M C-v 滚动其他窗格内容
C-x 4 f 在其他窗格中打开文件
C-x 4 0 关闭当前缓冲区和窗格
C-x 5 2 新建窗口(frame)
C-x 5 f 在新窗口中打开文件
C-x 5 o 切换至其他窗口
C-x 5 0 关闭当前窗口
\end{verbatim}
\vspace{-9pt}
\noindent \textbf{\xiaosi 书签命令(Bookmark  commands)} 
\vspace{-9pt}
\begin{verbatim}
C-x r m 在光标当前位置创建书签 
C-x r b 转到书签
M-x bookmark-rename 重命名书签
M-x bookmark-delete 删除书签
M-x bookmark-save 保存书签
C-x r l 列出书签清单
d 标记等待删除
Del 取消删除标记
x 删除被标记的书签
r 重命名
s 保存列表内所有书签
f 转到当前书签指向的位置
m 标记在多窗口中打开
v 显示被标记的书签(或者光标当前位置的书签)
t 切换是否显示路径列表
w 显示当前文件路径
q 退出书签列表

M-x bookmark-write 将所有书签导出至指定文件 
M-x bookmark-load 从指定文件导入书 签
\end{verbatim}
\vspace{-9pt}
\noindent \textbf{\xiaosi Shell}
\vspace{-9pt}
\begin{verbatim}
M-x shell 打开shell模式
C-c C-c 类似unix里的C-c(停止正在运行的程序)
C-d 删除光标后一个字符
C-c C-d 发送EOF

C-c C-z 挂起程序(unix下的C-z)
M-p 显示前一条命令
M-n 显示后一条命令
\end{verbatim}
\vspace{-9pt}
\noindent \textbf{\xiaosi DIRectory EDitor (dired)}
\vspace{-9pt}
\begin{verbatim}
C-x d 打开dired C(大写C) 复制
d 标记等待删除
D 立即删除
e或f 打开文件或目录
g 刷新当前目录
G 改变文件所属组(chgrp)
k 从屏幕上的列表里删除一行(不是真的删除)
m 用*标记
n 光标移动到下一行
o 在另一个窗格打开文件并移动光标
C-o  在另一个窗格打开文件但不移动光标
P 打印文件
q 退出dired
Q 在标记的文件中替换
R 重命名文件
u 移除标记
v 显示文件内容
x 删除有D标记的文件
Z 压缩/解压缩文件
M-Del 移除标记(默认为所有类型的标记)
~ 标记备份文件(文件名有~的文件)等待删除
# 标记自动保存文件(文件名形如#name#)等待删除
*/  用*标记所有文件夹(用C-u  */n移除标记)
= 将当前文件和标记文件(使用C-@标记而不是dired的m标记)比较
M-=  将当前文件和它的备份比较
!  对当前文件应用shell命令
M-}  移动光标至下一个用*或D标记的文件
M-{  移动光标至上一个用*或D标记的文件
\% d 使用正则表达式标记文件等待删除
\% m 使用正则表达式标记文件为*
+ 新建文件夹
>  移动光标至后一个文件夹
<  移动光标至前一个文件夹
s 切换排序模式(按文件名/日期)
或许把这个命令归入这一类也很合适: 
M-x speedbar 打开一个独立的目录显示窗口
\end{verbatim}
\vspace{-9pt}
\noindent \textbf{\xiaosi Telnet}
\vspace{-9pt}
\begin{verbatim}
M-x telnet 打开telnet模式
C-d 删除后一个字符或发送EOF
C-c C-c 停止正在运行的程序(和unix下的C-c类似)
C-c C-d 发送EOF
C-c C-o 清除最后一个命令的输出
C-c C-z 挂起正在运行的命令
C-c C-u 移除前一行
M-p 显示前一条命令
\end{verbatim}
\vspace{-9pt}
\noindent \textbf{\xiaosi Text}
\vspace{-9pt}
\begin{verbatim}
只能在text模式里使用
M-s 使当前行居中
M-S 使当前段落居中
M-x center-region 使被选中的区域居中
\end{verbatim}
\vspace{-9pt}
\noindent \textbf{\xiaosi 宏命令(Macro-commands)}
\vspace{-9pt}
\begin{verbatim}
C-x ( 开始定义宏
C-x ) 结束定义宏
C-x e 运行最近定义的宏
M-n C-x e 运行最近定义的宏n次
M-x name-last-kbd-macro 给最近定义的宏命名(用来保存)
M-x insert-kbd-macro 将己命名的宏保存到文件
M-x load-file 载入宏
\end{verbatim}
\vspace{-9pt}
\noindent \textbf{\xiaosi 编程(Programming)}
\vspace{-9pt}
\begin{verbatim}
M C-\ 自动缩进光标和标记间的区域
M-m 移动光标到行首第一个(非空格)字符
M-^  将当前行接到上一行末尾处
M-;  添加缩进并格式化的注释
C, C++和Java模式
M-a 移动光标到声明的开始处
M-e 移动光标到声明的结尾处
M C-a 移动光标到函数的开始处
M C-e 移动光标到函数的结尾处
C-c RETURN  将光标移动到函数的开始处并标记到结尾处
C-c C-q 根据缩进风格缩进整个函数
C-c C-a 切换自动换行功能
C-c C-d 一次性删除光标后的一串空格(greedy delete)
为了实现下面的一些技术,  你需要在保存源代码的目录里运行"etags
*.c *.h *.cpp"(或者源代码的其他的扩展名) 
M-.(点) 搜索标签
M-x tags-search ENTER 在所有标签里搜索(使用正则表达式)
M-,(逗号)   在tags-search里跳至下一个匹配处
M-x tags-query-replace 在设置过标签的所有文件里替换文本
\end{verbatim}
\vspace{-9pt}
\noindent \textbf{\xiaosi GDB(调武器)}
\vspace{-9pt}
\begin{verbatim}
M-x gdb 在另一个的窗格中打开gdb
\end{verbatim}
\vspace{-9pt}
\noindent \textbf{\xiaosi 版本控制(Version Control)}
\vspace{-9pt}
\begin{verbatim}
C-x v d 显示当前目录下所有注册过的文件
            (show all registered files in this dir) 
C-x v = 比较不同版本间的差异
            (show diff between versions)
C-x v u 移除上次提交之后的更改
            (remove all changes since last checkin)
C-x v ~ 在不同窗格中显示某个版本
            (show certain version in different window) 
C-x v l 打印日志(print log)
C-x v i 标记文件等待添加版本控制
           (mark file for version control add)
C-x v h 给文件添加版本控制文件头
            (insert version control header into file) 
C-x v r 获取命名过的快照
            (check out named snapshot)
C-x v s 创建命名的快照
            (create named snapshot)
C-x v a 创建gnu风格的更改日志
             (create changelog file in gnu-style)
\end{verbatim}
\end{multicols}
% Emacs 24.5.1 (Org mode 8.2.10)
\end{document}
